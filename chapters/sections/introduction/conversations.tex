\section{Which Conversations?}
    \subsection{Spotify Podcast Corpus \label{ssec: spotify corpus}}
        Conversation podcasts are a type of podcast in which the host invites one or more guests for a recorded conversation. The format is similar to a traditional talk-show, but conversations are usually longer (from 0.5--4 hours long), more in-depth, and follow a less rigid structure.  The recorded conversation is then published on streaming platforms such as YouTube or Spotify, where the audience can listen to it.
        
        The Spotify podcast corpus\cite{clifton-2020100000} is a collection of 100,000 episodes from different podcast shows hosted on Spotify and our primary source of conversations. It features over 50,000 hours of audio and 600 million words spoken in a large number of shows that feature a number of conversation lengths, topics, styles and qualities. The corpus provides audio files as well as podcast transcripts automatically transcribed by Google's speech to text service. The quality of these transcriptions varies heavily between topics discussed, microphone quality and clarity of the conversations and are a key limitation for our work.
        
    \subsection{Switchboard Dialogue Act Corpus \label{ssec: swda}}
        A secondary source of conversations is provided by the \gls{swda} corpus, a collection of 1155 five-minute conversations between two participants, in which callers question receivers on pre-determined topics such as child care, recycling and new media.\cite{fang2012annotation}. It is transcribed by humans, annotated with \glspl{da} (see Sec. \ref{ssec: DAs}) as well as the pre-determined topic of the conversation.
        %Besides being used as training data for some of the models we train, the \gls{swda} corpus provides us with a different type of conversation that we can analyse: short, private, topical phone conversations instead of in-depth conversations primarily recorded for an audience.