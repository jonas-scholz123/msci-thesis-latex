\section{Conversation Analysis \label{sec: ca}}
\glsreset{ca}
The analysis of the structure of conversations falls within the field of \gls{ca}, a sub-field of linguistics. This section explores the key concepts from \gls{ca} that we use for our work. 
% CONVERSATION ANALYSIS FIRST
    \subsection{Utterances \label{ssec: utterances}}
        Within conversations, \glspl{utterance} are strings of words that begin and end with a clear pause. As an approximation, we are assuming that every \gls{utterance} corresponds to a sentence and in this thesis, sentences and \glspl{utterance} are considered synonyms.
        
        
    \subsection{Dialogue Acts \label{ssec: DAs}}
        The atomic unit within \gls{ca} is the \gls{da}. The \gls{da} describes the \textit{social action} of an \gls{utterance}. \Glspl{da} can be split into categories such as questions, greetings, statements of politeness, statements of agreement etc. for our work, we use the SWBD-DAMSL tag-set of 42 \glspl{da} featured in the \gls{swda} corpus (see Sec. \ref{ssec: swda})\cite{fang2012annotation, swda}. The most common \glspl{da} as well as their relative frequency within the \gls{swda} corpus is shown in table \ref{table: damsl das}.
        
        \begin{table}[ht]
        \begin{tabular}{|l|l|l|}
        \hline
            \textbf{SWBD-DAMSL}          & \textbf{Example}                                & \textbf{\%} \\ \hline
            Statement-non-opinion        & Me, I'm in the legal department.                & 36\%        \\ \hline
            Acknowledge (Backchannel)    & Uh-huh.                                         & 19\%        \\ \hline
            Statement-opinion            & I think it's great                              & 13\%        \\ \hline
            Agree/Accept                 & That's exactly it.                              & 5\%         \\ \hline
            Abandoned or Turn-Exit       & So, -                                           & 5\%         \\ \hline
            Appreciation                 & I can imagine.                                  & 2\%         \\ \hline
            Yes-No-Question              & Do you have to have any special training?       & 2\%         \\ \hline
            Non-verbal                   & {[}Laughter{]}, {[}Throat\_clearing{]}          & 2\%         \\ \hline
            Yes answers                  & Yes.                                            & 1\%         \\ \hline
            Conventional-closing         & Well, it's been nice talking to you.            & 1\%         \\ \hline
            Uninterpretable              & But, uh, yeah                                   & 1\%         \\ \hline
            Wh-Question                  & Well, how old are you?                          & 1\%         \\ \hline
            No answers                   & No.                                             & 1\%         \\ \hline
            Response Acknowledgement     & Oh, okay.                                       & 1\%         \\ \hline
            Hedge                        & I don't know if I'm making any sense or not.    & 1\%         \\ \hline
            Declarative Yes-No-Question  & So you can afford to get a house?               & 1\%         \\ \hline
            Other                        & Well give me a break, you know.                 & 1\%         \\ \hline
            Backchannel in question form & Is that right?                                  & 1\%         \\ \hline
            Quotation                    & You can't be pregnant and have cats             & .5\%        \\ \hline
            Summarize/reformulate        & Oh, you mean you went home.                     & .5\%        \\ \hline
            Affirmative non-yes answers  & It is.                                          & .4\%        \\ \hline
        \end{tabular}
        \caption{The most common SWBD-DAMSL dialogue acts, taken from \cite{swda}. The third column is the fractional frequency of the DA within the \gls{swda} corpus of short conversations (see Sec. \ref{ssec: swda}).}
        \label{table: damsl das}
        \end{table} 
        

    
    %\subsection{Adjacency Pairs \label{ssec: adjacency pairs}}
        
        %In a conversation, speech acts are part of a greater cohesive structure. In \gls{ca}, this is the structure of \textit{adjacency pairs}. Adjacency pairs are pairs of speech acts $(s_j, s_{j+1})$, in which the initial speech act with label $s_j$ is replied to by the response with label $s_{j+1}$. Examples include:
        %\begin{itemize}
        %    \item (question, answer)
        %    \item (greeting, greeting)
        %    \item (request, denial)
        %    \item (opinion, agreement) etc.
        %\end{itemize}
        %A conversation, on the \gls{utterance} level, is structured as a long string of such adjacency pairs.
        
    
    