\section{An Overview of Machine Learning \label{sec: ML}}
    A traditional computer programme takes some inputs $X$, does something with those inputs and outputs some output $Y$. It is a function $f: X \rightarrow Y$. The behaviour of that function $f$ is defined by the programmer, who specifies what the computer is supposed to do with $X$ to turn it into $Y$.
    
    The objective of machine learning (ML) is to approximate that function $f$ as $\hat{f}$, not through implementation by a programmer, but by statistically learning from data. This function $\hat{f}$ is called the model. In practice, $\hat{f} = \hat{f}_\theta$ is highly flexible and highly dependent on a set of parameters $\theta$. These parameters are initially randomly assigned and then automatically tweaked until $\hat{f}_\theta$ shows the desired behaviour.
    
    ML is useful in cases that are too complicated to be implemented traditionally, with too many inputs and too many edge cases, such as image recognition or natural language understanding.
    
    \subsection{Supervised Learning}
        In supervised learning, the machine learning algorithm is fed pairs of inputs and matching outputs $(X, Y)$ that were labelled manually by humans. If, for example, one wanted to train a model that recognises cats, $X$ would be images of cats and images not of cats, and $Y$ would be the corresponding label \textit{cat} or \textit{not cat}. The algorithm uses $(X, Y)$ to iteratively compare its own prediction $\hat{f}_\theta(X) = \hat{Y}$ to the true output $Y$ and slightly adjusts its parameters $\theta$ to match that true output. This process is called training and the pairs of $(X, Y)$ accordingly are called training data. There are many different approaches and implementations of the functions $\hat{f}_\theta$ and the training process. The type of $\hat{f}_\theta$ we use is called a neural network (NN) and it is trained through a process called backpropagation. We expand on this in section ???. Supervised learning is the main flavour of ML used for this project.
        
    \subsection{Unsupervised Learning}
        In unsupervised learning, only the input data $X$ is provided (and no true labels $Y$). The ML algorithm then uses some statistical techniques to provide some useful insights. The two main uses of unsupervised learning in this project are principal component analysis (PCA), which reduces high-dimensional data to lower-dimensional data while trying to preserve similarity, and cluster analysis, which finds similarities within the datapoints $X$ and groups similar data-points together.