\section{An Overview of Natural Language Processing \label{sec: nlp}}

Natural Language Processing (NLP) is a sub-field of computer science, artificial intelligence and linguistics, which attempts to use computers to analyse natural (human) language data. An example of advanced NLP applications are \textit{virtual assistants} such as the \textit{Google Home} or \textit{Amazon Alexa} devices, which need to understand queries spoken by a human (speech recognition), understand them (natural language understanding), execute the functionality that was asked for and formulate a response (natural language generation). The methods we apply fall firmly within the area of natural language understanding, as we apply (and modify) commonly used methods to the largely unexplored landscape of human conversations.

%NLP has had an increasingly important role and is commonly referenced as a key component to an \textit{artificial general intelligence} (AGI). CITE TURING.

Most NLP applications deal with low-noise text data, such as news articles, product reviews, or direct queries. Conversations between humans do not fall within this category and are associated with large amounts of noise: from members of the conversation talking over one-another to the use of sarcasm, jokes or slang, conversations are messy. For this reason, conversation analysis has largely been manual in nature.