\makeglossaries
\newacronym{lccrf}{LCCRF}{Linear Chain Conditional Random Field}


\newglossaryentry{nlp}
{
        name=NLP,
        description={Natural Language Processing. The field of computational linguistics concerned with allowing computers to process and analyse natural (human) language data, specifically tasks that require the programme to ``understand" the text in some way},
        first = {Natural Language Processing (NLP)},
        long = {Natural Language Processing}
}

\newglossaryentry{crf}
{
        name=CRF,
        description={Conditional Random Field. A graph-based modelling technique that classifies sequences of tags given a sequence of inputs.},
        first = {Conditional Random Field (CRF)},
        long = {Conditional Random Field},
        plural = {CRFs},
        firstplural = {Conditional Random Fields (CRFs)},
}

\newglossaryentry{ml}
{
        name=ML,
        description={Machine Learning. The area of computer science concerned with learning the behaviour of a computer by examples instead of implementation by a programmer.},
        first = {Machine Learning (ML)},
        long = {Machine Learning}
}

\newglossaryentry{ca}
{
        name=CA,
        description={Conversation Analysis. A sub-field of linguistics concerned with the study of conversations.},
        first = {Conversation Analysis (CA)},
        long = {Conversation Analysis}
}

\newglossaryentry{lstm}
{
        name=LSTM,
        description={Long short term memory. A type of RNN architecture},
        first = {Long Short Term Memory (LSTM)},
        long = {Long Short Term Memory}
}

\newglossaryentry{geek}
{
        name=GEEK,
        description={Graph of Embedded Extracted Keyphrases. Our novel topic extraction algorithm},
        first = {Graph of Embedded Extracted Keyphrases (GEEK)},
        long = {Graph of Embedded Extracted Keyphrases}
}

\newglossaryentry{gru}
{
        name=GRU,
        description={Gated Recurrent Unit. A type of RNN architecture},
        first = {Gated Recurrent Unit (GRU)},
        long = {Gated Recurrent Unit}
}


\newglossaryentry{rnn}
{
        name=RNN,
        description={Recurrent Neural Network. Neural networks that can process sequences of varying length by passing a state of numbers from the first to the last neuron in the layer, modifying the state in a way depending on the input at the corresponding position within the sequence},
        first = {Recurrent Neural Network (RNN)},
        long = {Recurrent Neural Network},
        plural = {RNNs},
        firstplural = {Recurrent Neural Networks (RNNs)},
}

\newglossaryentry{nn}
{
        name=NN,
        description={(Artificial) Neural Network. A very flexible function (loosely based on the neurons of human brains) whose behaviour is specified by a set of parameters. A type of machine learning model},
        first = {Neural Network (NN)},
        long = {Neural Network},
        plural = {NNs},
        firstplural = {Neural Networks (NNs)},
}

\newglossaryentry{swda}
{
        name=SwDA,
        description={Switchboard Dialogue Act. The name of a dataset of phone conversations held via the Switchboard helpline telephone service annotated with dialogue acts by humans},
        first = {Switchboard Dialogue Act (SwDA)},
        long = {Switchboard Dialogue Act},
}

\newglossaryentry{lda}
{
        name=LDA,
        description={Latent Dirichlet Allocation. A topic model that assumes text is generated by repeatedly sampling a distribution of topics and then sampling that topic for a word.},
        first = {Latent Dirichlet Allocation (LDA)},
        long = {Latent Dirichlet Allocation},
}

\newglossaryentry{da}
{
        name=DA,
        description={Dialogue Act. The intended social action of an utterance. E.g. the intended social action of the utterance ``Hello" is a greeting},
        first = {Dialogue Act (DA)},
        long = {Dialogue Act},
        plural = {DAs},
        firstplural = {Dialogue Acts (DAs)},
}

\newglossaryentry{pca}
{
        name=PCA,
        description={Principal Component Analysis. A dimensionality reduction technique that determines convenient basis vectors and maps data onto these basis vectors.},
        first = {Principal Component Analysis (PCA)},
        long = {Principal Component Analysis},
}

\newglossaryentry{pos}
{
        name=PoS,
        description={Part of Speech. The grammatical function a word serves, e.g. ``house" $\rightarrow$ noun},
        first = {Part of Speech (PoS)},
        long = {Part of Speech},
}

\newglossaryentry{ner}
{
        name=NER,
        description={Named Entity Recognition. Identifying named entities, such as ``Donald Trump", ``CIA" or, ``Coca-Cola" automatically},
        first = {Named Entity Recognition (NER)},
        long = {Named Entity Recognition},
}


\newglossaryentry{embedding}{
    name=embedding,
    description={A vector representing the meaning of a word or of a sequence of words}
    }

\newglossaryentry{model}{
    name=model,
    description={A function mapping inputs to outputs whose behaviour is statistically learned given a set of training data}
    }

\newglossaryentry{neuron}{
    name=neuron,
    description={An elementary information processing device that takes an input, and outputs a normalised weighted sum. The weights in the weighted sum determine its behaviour and are statistically learned. The atomic components of neural networks}
    }

\newglossaryentry{glove}{
    name=GloVe,
    description={\textbf{Glo}bal \textbf{Ve}ctors. A type of word embedding}
    }

\newglossaryentry{numberbatch}{
    name=ConceptNet Numberbatch,
    description={A type of word embedding}
    }

\newglossaryentry{utterance}{
    name=utterance,
    description={A section of spoken language that begins and ends with a pause from the speaker. In this work we approximate utterances as sentences}
    }

\newglossaryentry{keyphrase}{
    name=key-phrase,
    description={Single or multi-word expressions that represent the topics of a text}
    }
